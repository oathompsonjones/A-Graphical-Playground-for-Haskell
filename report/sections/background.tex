% Define the document as a subfile and reference its root.
\documentclass[main.tex]{subfiles}

% Begin the document.
\begin{document}
\chapter{Background}
    % \paragraph{What is required?}
    %     You should aim to produce around 5 pages of synopsis of the existing work that
    %         forms the context for your project.
    %     You should give this to your supervisor by the end of week 5, and discuss it
    %         with them at your next meeting, jointly filling in and submitting this form.

    %     It is suggested that you use this as an opportunity to familiarise yourself
    %         with LaTeX, using the report template shown in Style Guide provided.
    %     The University has reasonable guides for beginners on how to use LaTeX with or
    %         without overleaf.

    %     Things to consider:
    %     \begin{itemize}
    %         \item Although typically published papers will be a substantial part of the background
    %               (see more below) this may differ for different projects.
    %               For example, your task may instead require you to get a strong understanding of
    %                   an existing code-base, or critical software tools, or of an application domain.
    %               You will still need to demonstrate your knowledge and critical evaluation
    %                   through writing a good synopsis of the relevant existing material, to set your
    %                   work in context, and justify your approach.

    %         \item A good background chapter is not just a summary of all the things you have read.
    %               Rather, it should be selective and well-organised, to tell the reader just what
    %                   they need to know to understand the work that follows.

    %         \item It should be clear to the reader (and to you!) why you are including any information
    %               that you present.
    %               A lengthy, rambling and ‘padded’ background chapter is a common reason for
    %                   lower marks on these criteria; or a chapter that omits critical material.

    %         \item Probably the best guide to the expected content is to look at published papers in
    %               the same domain as your project, and how they put their new work in the context of
    %               previous work.
    %     \end{itemize}

\end{document}

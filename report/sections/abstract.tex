% Define the document as a subfile and reference its root.
\documentclass[main.tex]{subfiles}

% Begin the document.
\begin{document}
\section*{Abstract}
    \paragraph{}
        Haskell~\cite{haskell} is a general-purpose, statically-typed, purely
            functional programming language with type inference and lazy
            evaluation~\cite{wiki-haskell}.

        \begin{listing}[H]
            \caption{A simple Haskell Hello World program.}
            \begin{minted}{haskell}
                -- A simple Haskell Hello World program.
                
                main :: IO ()
                main = putStrLn "Hello, World!"
            \end{minted}
            \label{listing:hello-world}
        \end{listing}

        Listing~\ref{listing:hello-world} shows a simple Haskell program that prints
            \verb|"Hello, World!"| to the console.

        \begin{figure}[H]
            \centering
            \includegraphics[width=0.5\textwidth]{haskell.jpg}
            \caption{The Haskell logo.}
            \label{fig:haskell-logo}
        \end{figure}

        \begin{figure}[H]
            \centering
            \begin{tikzpicture}[]
                % Nodes
                \node[] (a) [] {$a$};
                \node[] (b) [right = of a] {$b$};
                \node[] (d) [below = of a] {$d$};
                \node[] (c) [left  = of d] {$c$};
                \node[] (e) [right = of d] {$e$};
                \node[] (f) [right = of e] {$f$};
                \node[] (g) [below = of d] {$g$};
                \node[] (h) [right = of g] {$h$};

                % Edges 
                \draw[] (a) -- (b);
                \draw[] (a) -- (c);
                \draw[] (a) -- (d);
                \draw[] (a) -- (e);
                \draw[] (b) -- (d);
                \draw[] (b) -- (e);
                \draw[] (b) -- (f);
                \draw[] (c) -- (d);
                \draw[] (c) -- (g);
                \draw[] (d) -- (e);
                \draw[] (d) -- (g);
                \draw[] (d) -- (h);
                \draw[] (e) -- (f);
                \draw[] (e) -- (g);
                \draw[] (e) -- (h);
                \draw[] (f) -- (h);
                \draw[] (g) -- (h);
            \end{tikzpicture}
            \caption{This is a diagram.}
            \label{fig:diagram}
        \end{figure}

\end{document}

% Define the document as a subfile and reference its root.
\documentclass[../main.tex]{subfiles}

% Begin the document.
\begin{document}
\chapter{Introduction} \label{ch:introduction}
    This project — A Graphical Playground for Haskell — is a web-based Haskell
        editor with a built-in graphics library, allowing users to create images and
        animations using Haskell, and render them directly in the browser.
    The editor provides a simple, interactive interface for writing Haskell code
        and viewing the output in real-time, all within a web browser.
    The project creates a beginner-friendly experience by providing a visual,
        interactive environment for learning and experimenting with functional
        programming, while also offering a fun and engaging way to create graphics and
        animations for users of all skill levels.

    \section{Aims and Motivations}
        There are a variety of motivations for creating this project, including
            educational benefits for students learning Haskell, as well as practical
            applications for users interested in creating graphics and animations.

        The project aims to provide an experience that is both beginner-friendly and
            engaging for users of all skill levels.
        For beginners, the editor offers a visual, interactive environment for learning
            Haskell, allowing them to see how their code works in real-time.
        For more experienced users, the editor provides a fun and engaging way to
            create graphics and animations using Haskell, with the ability to experiment
            and iterate quickly.

        \subsection{Educational Value}
            Imagine a first-year Computer Science class — a mix of students with varying
                levels of programming experience, from those who have never written a line of
                code to those who have been programming for years.
            Those who do have experience programming have mostly worked with imperative
                languages like Python or Java, and are now learning Haskell for the first time.
            The students follow along, trying to understand the code and how it works; but
                as the program grows more complex, many students start to get lost in the world
                of functional programming.
            Without a way to visualise the program's execution or to see how the different
                parts of the program interact, they struggle to keep up with the lectures.
            They have difficulty seeing how different parts of the program fit together, or
                how the program's output changes as they modify the code.

            This project aims to address this problem by providing students with an
                interactive, visual environment for learning Haskell, and by extension,
                functional programming.
            A graphical editor allows students to see how their code works in real-time,
                helping them understand the program's execution and how the different parts of
                the program interact, while also providing a fun and engaging way to learn
                Haskell.
            By providing a web-based service, users never even need to install Haskell on
                their own computers — they can start coding right away, from any device with a
                web browser.

        \subsection{Practical Applications}
            Beyond its educational value, the project also has practical applications for
                users interested in creating graphics and animations using Haskell.
            The editor's simple interface and built-in graphics library make it easy to
                create images and animations, with the ability to experiment and iterate
                quickly.
            By including documentation for the graphics library and examples of how to use
                it, the project aims to make it easy for users to get started creating their
                own graphics and animations.

    \section{Dissertation Outline}
        Chapter~\ref{ch:background} provides relevant background information on
            Haskell, functional programming, graphics, and web technologies.
        This is followed by an exploration of related works, placing this project in
            context of existing Haskell editors and graphics libraries.

        Chapter~\ref{ch:graphics} describes the design and implementation of the
            graphics library, and the challenges faced in creating it.
        It compares alternative design choices, explaining the rationale behind the
            decisions made, and the implementation of these choices.

        In Chapter~\ref{ch:website}, the design and implementation of the website are
            detailed, illustrating the technologies used and the development process,
            alongside any challenges faced.
        This includes the development of both the front-end and back-end components of
            the website, as well as their integration and deployment.

        Chapters \ref{ch:graphics} and \ref{ch:website} can be read in any order, as
            they are largely independent of one another.
        The placement of the graphics library before the website provides a more
            logical progression of the project's development, starting from core
            functionality, building to the final product.

        Chapter~\ref{ch:evaluation} evaluates the project how successfully the project
            meets its aims and objectives, as well as how well it performs in practice.
        This includes a discussion of the user testing and feedback survey conducted,
            as well as any insights gained from the evaluation process.

        Finally, Chapter~\ref{ch:conclusions} concludes the dissertation, summarising
            the project, reflecting on the process of creating it, and discussing the
            potential impact of the project.
        It discusses the potential areas for future work and improvements to the
            project, including suggestions for how the project could be extended in the
            future, as well as any limitations of the current implementation.

\end{document}

\documentclass[../main.tex]{subfiles}

\begin{document}
\chapter{Testing and Evaluation} \label{ch:evaluation}
    Testing a website is not quite as simple as testing other software.
    Websites are not standalone applications, but rather a collection of files that
        are interpreted by a web browser.
    These browsers can vary greatly in their implementation of web standards, and
        the same website can look and behave differently in different browsers.
    In particular, CSS can be a source of many problems.
    This chapter will discuss the testing methodology used to ensure that the
        website is functional and accessible to all users.

    \section{Testing Methodology}
        \subsection{Browser and Device Compatibility}
            To ensure compatibility between browsers, the website was tested in the
            following browsers:
            \begin{itemize}
                \item Google Chrome
                \item Mozilla Firefox
                \item Safari
            \end{itemize}
            The website was also tested on devices with different screen sizes to ensure
                that the website is responsive.
            The website was tested on the following devices:
            \begin{itemize}
                \item 28-inch monitor
                \item 14-inch laptop
                \item 11-inch tablet
                \item 6.7-inch smartphone
                \item 5.85-inch smartphone
            \end{itemize}
            To test the transitions between these sizes, the website was tested using
                ``Responsive Design Mode'' in Safari.

        \subsection{CSS and JavaScript Robustness}
            As CSS and JavaScript are the backbone of the website, it is important to
                ensure that they are robust.
            Fortunately, the Material UI library is well-tested and widely used, so many
                components are already tested.
            However, the custom CSS and JavaScript code must be tested.
            Components which display variable amounts of data, such as the code editor and
                the console output, were tested with different amounts of data to ensure that
                they can handle large amounts of data without breaking.

        \subsection{The Graphics Library}
            The graphics library was tested by creating a series of programs, covering
                every feature of the library.
            These programs were then run on the website to ensure that they worked as
                expected.
            The programs were also run with different input values to ensure that the
                library can handle a wide range of inputs.
            The results of these tests were then compared to the expected behaviour, as
                documented in the reference page, to ensure that these match.

    \section{Testing Results}
        On the whole, the website performed well in testing.
        The website was responsive on all devices tested, and the transitions between
            different screen sizes were smooth.
        The majority of the web browsers tested, displayed the website correctly.

        There were some exceptions, however.

        \subsection{Resizable Areas}
            When the console output exceeds the size of the console, the console becomes
                scrollable, and the page should not resize.
            This worked as expected, until the content of the console area became taller
                than the height of its parent container (that being the total height of both
                the code editor and the console).
            At this point, the entire page would resize to accommodate the console, which
                was not the desired behaviour.

            This was caused by an error in the calculation of the size of each area.
            This calculation was done using pixel values, which are fixed, rather than
                percentages, which are relative.
            Changing the calculation to use percentages fixed this issue.

        \subsection{Scrolling in Firefox}
            The program examples on the homepage were designed to scroll to the middle
                automatically when the page is loaded.
            While this worked perfectly on other browsers, allowing users to scroll in both
                directions, Firefox did not allow users to scroll left.
            Instead, the contents of the scrollable area were correctly centred, but the
                elements which were now off-screen to the left could not be accessed.
            This particular error was rectified by using some extra JavaScript to set the
                correct scroll position after the page had loaded.

        \subsection{The Embedded Reference Page}
            The embedded reference page on the editor page caused two issues, one in all
                browsers, and one in Firefox.

            The Firefox error was once again related to scrolling.
            Switching to the reference tab would resize the page to accommodate the
                reference page, instead of making just the right column scrollable.
            This was fixed by adding a calculation in CSS to ensure that the reference page
                could not exceed the height of the parent container.

            The second issue allowed the components of the reference page to exceed the
                width of their parent container.
            A simple fix for this was to add a new \texttt{<div>} element around the
                reference page, and set the width of this element to 100\%.

        \subsection{The Graphics Library}
            For the most part, the graphics library worked as expected.
            One issue that was encountered, however, was a limitation of working with
                infinite lists.
            If the user tires to render an animation with an infinite number of frames, the
                browser quickly drops in performance, becoming unresponsive unless refreshed.
            This is a limitation of how the website is implemented, and not a limitation of
                the graphics library itself.
            This limitation was documented in the reference page.
            The site will automatically loop the animation, so users can still create the
                appearance of infinite animations by using a finite number of frames, and
                having the first and last frames of the animation be the same.

    \section{User Testing and Feedback}
        A survey was created to test how users interacted with the website.
        The survey asked users to complete a short series of tasks including:
        \begin{itemize}
            \item Writing and running a basic Haskell program, without graphics.
            \item Learning how to produce a basic graphic by using the reference page.
            \item Altering a pre-existing program to produce a different image.
            \item Altering a pre-existing program to produce a different animation.
        \end{itemize}

        Participants were asked to rate the ease of each task from one to ten, with one
            being very difficult and ten being very easy, and to state roughly how long
            each task took.
        They were also asked to provide any feedback they had on the website, both
            positive and negative, and to identify any bugs they encountered.
        Participants were kept anonymous, but were asked to describe their
            qualifications in Computer Science, and their experience with Haskell.

        \subsection{Survey Responses}
            The survey was shared among Informatics students at the University of
                Edinburgh, and was posted on the Haskell subreddit.
            The survey received a total of 11 responses, alongside additional feedback from
                members of the subreddit.

            \subsubsection{User Experience and Qualifications}
                Three of the participants have the equivalent of (or are currently studying
                    for) an A-Level in Computer Science.
                Two of these participants described their experience with Haskell as
                    ``beginner'', while the third described their experience as ``intermediate''.

                Five of the participants have the equivalent of (or are currently studying for)
                    a bachelor's degree in Computer Science.
                Four of these participants described their experience with Haskell as
                    ``intermediate'', while the fifth described their experience as ``beginner''.

                Two participants have the equivalent of (or are currently studying for) a
                    postgraduate degree in Computer Science, both of whom described their
                    experience with Haskell as ``advanced''.

                The final participant described their Computer Science qualifications as
                    ``other'', saying ``Every day is a school day'', and their experience with
                    Haskell as ``intermediate''.

                Fully tabulated results for all quantitative data can be found in
                    Appendix~\ref{app:feedback}.
                A summary of the results is as follows:

            \subsubsection{Writing a Basic Haskell Program}
                \begin{table}[H]
                    \centering
                    \begin{tabular}{c|c|c}
                        \textbf{Category} & \textbf{Average Ease of Task} & \textbf{Average Time Taken} \\
                        \hline
                        A-Level           & 2                             & 12.67-16.67 minutes         \\
                        Bachelor's        & 7.4                           & 2-6 minutes                 \\
                        Postgraduate      & 10                            & less than a minute          \\
                        Other             & 10                            & 1-5 minutes                 \\
                        \hline
                        Beginner          & 2.67                          & 11-15 minutes               \\
                        Intermediate      & 7.5                           & 2.67-6.67 minutes           \\
                        Advanced          & 10                            & less than a minute          \\
                        \hline
                        Overall           & 6.64                          & 4.45-7.72 minutes           \\
                    \end{tabular}
                    \caption{The average results of the first task.}
                \end{table}

                These results indicate show that the writing and executing a basic Haskell
                    program through the website was straightforward for those with experience in
                    Haskell, but quite difficult for those without.
                This indicates that the website is quite intuitive to use, but that it could
                    provide more documentation for Haskell itself, rather than giving a brief
                    summary, and linking to external resources.

            \subsubsection{Producing a Simple Graphic Using the Reference Page}
                \begin{table}[H]
                    \centering
                    \begin{tabular}{c|c|c|c}
                        \textbf{Category} & \textbf{Average Ease of Task} & \textbf{Average Quality of Docs} & \textbf{Average Time Taken} \\
                        \hline
                        A-Level           & 2.33                          & 3.33                             & 9.33-13.33 minutes          \\
                        Bachelor's        & 6                             & 6.6                              & 8-12 minutes                \\
                        Postgraduate      & 7.5                           & 7.5                              & 6-10 minutes                \\
                        Other             & 8                             & 7                                & 1-5 minutes                 \\
                        \hline
                        Beginner          & 3.33                          & 5                                & 11-15 minutes               \\
                        Intermediate      & 5.83                          & 5.83                             & 6-10 minutes                \\
                        Advanced          & 7.5                           & 7.5                              & 6-10 minutes                \\
                        \hline
                        Overall           & 5.45                          & 5.91                             & 7.36-11.36 minutes          \\
                    \end{tabular}
                    \caption{The average results of the second task.}
                \end{table}

                These results demonstrate that those with more experience both in the field of
                    Computer Science and in Haskell found the documentation to be of reasonable
                    high quality, while those with less experience did not.
                This could be a result of participants with less experience having less
                    experience with reading documentation in general, but would also suggest that,
                    as before, the website could benefit from more detailed documentation for
                    beginners.

                As to be expected, more experienced users once again found this task easier
                    than those with less experience.
                Interestingly though, users with less experience found this task easier than
                    the previous task, whereas more experienced users found this task harder.
                As users should have learnt how to write basic Haskell programs in the previous
                    task, they should no longer be at a disadvantage in this task.
                This would suggest, that once users have worked out how to write a basic
                    Haskell program, they are able to use the reference page to produce simple
                    graphics with relative ease, albeit still with some difficulty.

            \subsubsection{Altering a Pre-existing Program to Produce a Different Image}
                \begin{table}[H]
                    \centering
                    \begin{tabular}{c|c|c}
                        \textbf{Category} & \textbf{Average Ease of Task} & \textbf{Average Time Taken} \\
                        \hline
                        A-Level           & 9.67                          & 0.67-3.33 minutes           \\
                        Bachelor's        & 9                             & 0.8-4 minutes               \\
                        Postgraduate      & 5.33                          & 3.5-7.5 minutes             \\
                        Other             & 10                            & 1-5 minutes                 \\
                        \hline
                        Beginner          & 9.67                          & 0.67-3.33 minutes           \\
                        Intermediate      & 9.17                          & 0.83-4.17 minutes           \\
                        Advanced          & 5.33                          & 3.5-7.5 minutes             \\
                        \hline
                        Overall           & 9.09                          & 1.27-4.54 minutes           \\
                    \end{tabular}
                    \caption{The average results of the third task.}
                \end{table}

                These results show that altering a pre-existing program to produce a different
                    image was easy for most users, regardless of their experience.
                This suggests that the graphics library is easy to grasp, if provided with an
                    example to work from.
                Feedback from the postgraduate participant who gave a lower score for this task
                    suggests that they accomplished the task to a degree, but were not happy with
                    their result, as it did not match the provided example closely enough.

            \subsubsection{Altering a Pre-existing Program to Produce a Different Animation}
                \begin{table}[H]
                    \centering
                    \begin{tabular}{c|c|c}
                        \textbf{Category} & \textbf{Average Ease of Task} & \textbf{Average Time Taken} \\
                        \hline
                        A-Level           & 0.67                          & 11+ minutes                 \\
                        Bachelor's        & 7                             & 4-8 minutes                 \\
                        Postgraduate      & 7.5                           & 3.5-7.5 minutes             \\
                        Other             & 10                            & 1-5 minutes                 \\
                        \hline
                        Beginner          & 3                             & 12.67+ minutes              \\
                        Intermediate      & 6.33                          & 2.67-6.67 minutes           \\
                        Advanced          & 7.5                           & 3.5-7.5 minutes             \\
                        \hline
                        Overall           & 5.63                          & 5.55+ minutes               \\
                    \end{tabular}
                    \caption{The average results of the fourth task.}
                \end{table}

                The results for this task indicate a substantial leap in complexity from the
                    previous task.
                A-Level and beginner participants in particular found this task to be
                    difficult, with one A-Level participant taking over 21 minutes to complete the
                    task.
                This suggests that the website could benefit from more detailed documentation
                    on how to move from images to animations.
                For more experienced participants, this task was still more difficult than the
                    previous task, but not to the same extent.
                These participants were able to complete the task in a reasonable amount of
                    time, and with a reasonable level of ease.

            \subsubsection{Overall Ratings}
                \begin{table}[H]
                    \centering
                    \begin{tabular}{c|c}
                        \textbf{Category} & \textbf{Average Rating} \\
                        \hline
                        A-Level           & 3                       \\
                        Bachelor's        & 4.4                     \\
                        Postgraduate      & 4.5                     \\
                        Other             & 5                       \\
                        \hline
                        Beginner          & 3.33                    \\
                        Intermediate      & 4.33                    \\
                        Advanced          & 4.5                     \\
                        \hline
                        Overall           & 4.09                    \\
                    \end{tabular}
                    \caption{The average overall ratings of the website, out of 5.}
                \end{table}

                The overall ratings for the website were generally positive, with the majority
                    of participants rating the website as above average.
                The website was rated most highly by those with more experience, and least
                    highly by those with less experience.
                This once again indicates that the website provides a better user experience
                    for users with at least bachelor's level experience in Computer Science, and
                    some experience with Haskell.

            \subsubsection{Feedback}
                The feedback received from participants was generally positive, however there
                    were a few areas for improvement that were highlighted.
                Several participants indicated initial difficulties due to their lack of
                    experience with Haskell, but found the following tasks to be easier, once they
                    had a better understanding of the language.
                A few participants found the reference page lacked clear instructions for how
                    to render a shape once it had been defined.
                The documentation has since been updated to make this clearer.

                Participants praised the website's clean design, ease-of-use and speed.
                Many spoke highly of the graphics library, and the example image provided in
                    the reference page, which helped them to understand how to use the library.
                The examples on the homepage were described by multiple participants as
                    ``interesting'', ``useful'', ``informative'', ``entertaining'' and even
                    ``incredible'' by one participant.

        \subsection{Other Feedback from Users}

            Alongside the survey responses, feedback was also received from users on the
                Haskell subreddit.
            This feedback was extremely positive, with one user saying,
            \begin{aquote}{u/mlitchard}
                This an important step forward in the production of educational artefacts for
                    Haskell.
            \end{aquote}
            and another saying of the reference page,
            \begin{aquote}{u/CubOfJudahsLion}
                This is the level of friendliness in API docs that would get coders to come
                    back to Haskell.
            \end{aquote}
            This feedback is greatly encouraging, and suggests that users could find the
                website to be a useful tool for learning Haskell.

            Several users drew comparisons to other Haskell graphics tools, including
                Diagrams and Haskell for Mac.
            One user compared this project to CodeWorld, which led to the creator of
                CodeWorld commenting on the post.
            His assessment of the project was as follows:
            \begin{aquote}{u/cdsmith}
                \begin{itemize}
                    \item As far as overall goal, it looks like this is
                          staking out a path that's solidly and consistently using simple but normal
                          Haskell.
                          This is different from CodeWorld, which offers both an educational dialect
                              that's aggressively monomorphized, uncurried, and otherwise removes any
                              overloading type features that cause problems for beginners, but then also
                              offers a Haskell mode with a more conventional Haskell API.
                    \item In terms of implementation, this project seems to run code on the server and
                          stream frames of drawing instructions back to the client to be interpreted in
                          TypeScript.
                          CodeWorld, by contrast, compiles Haskell code into JavaScript that runs in the
                              web browser directly.
                    \item The API here is a bit less declarative in flavor than CodeWorld's.
                          For instance, while CodeWorld works very hard to make its \texttt{Picture} type
                              denotationally equivalent to \texttt{Point -> Color}, the analogous
                              \texttt{Shape} type here appears to be rather more complex, including explicit
                              notions of path, stroke, fill, etc. that mirror JavaScript's canvas API.
                          That does give it some more flexibility, but at the cost of a more complex
                              abstraction.
                    \item There does not appear to be an API for interactive or stateful programs here.
                          (Not surprising, since it's a much newer project).
                \end{itemize}
            \end{aquote}
            This feedback is particularly useful, as it provides a comparison between this
                project and CodeWorld from the creator of CodeWorld himself, and highlights
                some key differences between the two projects.

\end{document}

% Define the document as a subfile and reference its root.
\documentclass[../main.tex]{subfiles}

% Begin the document.
\begin{document}
\chapter{Conclusions} \label{ch:conclusions}
    Overall, the project has been a success.
    The website is functional, and the graphics library is capable of producing a
        wide range of images and animations.
    The website works well on a range of devices, and web browsers, is easy to use,
        and provides a good user experience.
    The feedback from users has been positive, and the website has been
        well-received by members of the Haskell community.

    \section{Future Work}
        While the system successfully achieves its goals, there is always room for
            improvement.
        A number of potential expansions could be made to the system, both in the
            graphics library and the website.
        Such additions could allow users to make better graphics, and provide an even
            better user experience.

        \subsection{The Graphics Library}
            Firstly, the library could be expanded to include more shapes and
                transformations, such as shearing and reflecting, allowing for more complex
                graphics to be created more easily.
            Secondly, the library could venture into functional reactive programming,
                allowing for more complex animations, which could be controlled by the user
                while the animation is running.
            Interactive programs were considered too complex for the scope of this project,
                but could be an extremely powerful addition to both the library and the
                website.

        \subsection{The Website}
            Modern IDEs provide a plethora of features to help users write code more
                efficiently.
            While the website's editor provides syntax highlighting, it could be expanded
                to include more features, such as inline error checking and intellisense.
            These features would allow users to find bugs before executing their programs,
                and make it easier to write their programs, without needing to refer to the
                documentation as frequently.

            While the system currently allows users to save their programs to a database,
                it could prove beneficial to allow users to connect their GitHub accounts, and
                save their programs to a GitHub repository.
            This would allow users to easily share their programs with others, while also
                providing incremental backups for their work.
            Additionally, GitHub integration could go further, allowing users with access
                to GitHub Copilot to use its suggestions directly in the editor.

\end{document}
